\RequirePackage{pdfmanagement-testphase}
\DeclareDocumentMetadata {lang=en-US}

% xmp metadata for pdf
% Originally used \usepackage[a-2a]{pdfx}
% \usepackage{hyperxmp} replaced it
% \RequirePackage{pdfmanagement-testphase} replaced it
% \PassOptionsToPackage{enable-debug,check-declarations}{expl3} broke with version 0.9 of tagpdf
% \ExplSyntaxOn no need for these 3 lines because metadata can handle it
% \pdfmanagement_add:nnn{Catalog}{Lang}{(enUS)} enUS is wrong, should be en-US
% \ExplSyntaxOff

\documentclass[11pt,
  english,
  a4paper,
]{article}
\usepackage{sa4ss}
\usepackage{amsmath,amssymb,array}
\usepackage{booktabs}

% From tagged-template.latex
\usepackage{lmodern}
\usepackage{ifxetex,ifluatex}
\ifnum 0\ifxetex 1\fi\ifluatex 1\fi=0 % if pdftex
  \usepackage[T1]{fontenc}
  \usepackage[utf8]{inputenc}
  \usepackage{textcomp} % provide euro and other symbols
\else % if luatex or xetex
  \usepackage{unicode-math}
  \defaultfontfeatures{Scale=MatchLowercase}
  \defaultfontfeatures[\rmfamily]{Ligatures=TeX,Scale=1}
\fi

% Use upquote if available, for straight quotes in verbatim environments
\IfFileExists{upquote.sty}{\usepackage{upquote}}{}
\IfFileExists{microtype.sty}{% use microtype if available
  \usepackage[]{microtype}
  \UseMicrotypeSet[protrusion]{basicmath} % disable protrusion for tt fonts
}{}
\makeatletter
\@ifundefined{KOMAClassName}{% if non-KOMA class
  \IfFileExists{parskip.sty}{%
    \usepackage{parskip}
  }{% else
    \setlength{\parindent}{0pt}
    \setlength{\parskip}{6pt plus 2pt minus 1pt}}
}{% if KOMA class
  \KOMAoptions{parskip=half}}
\makeatother
\usepackage{xcolor}
\IfFileExists{xurl.sty}{\usepackage{xurl}}{} % add URL line breaks if available
\hypersetup{
  pdftitle={Proposed decision tables for copper rockfish (Sebastes caurinus) off the U.S. West Coast},
  pdflang={en},
  hidelinks,
  pdfcreator={LaTeX via pandoc}}
\urlstyle{same} % disable monospaced font for URLs
\usepackage{longtable}
% Correct order of tables after \paragraph or \subparagraph
\usepackage{etoolbox}
\makeatletter
\patchcmd\longtable{\par}{\if@noskipsec\mbox{}\fi\par}{}{}
\makeatother
% Allow footnotes in longtable head/foot
\IfFileExists{footnotehyper.sty}{\usepackage{footnotehyper}}{\usepackage{footnote}}
\makesavenoteenv{longtable}
\usepackage{graphicx}
\makeatletter
\def\maxwidth{\ifdim\Gin@nat@width>\linewidth\linewidth\else\Gin@nat@width\fi}
\def\maxheight{\ifdim\Gin@nat@height>\textheight\textheight\else\Gin@nat@height\fi}
\makeatother
% Scale images if necessary, so that they will not overflow the page
% margins by default, and it is still possible to overwrite the defaults
% using explicit options in \includegraphics[width, height, ...]{}
\setkeys{Gin}{width=\maxwidth,height=\maxheight,keepaspectratio}
% Set default figure placement to htbp
\makeatletter
\def\fps@figure{htbp}
\makeatother
\setlength{\emergencystretch}{3em} % prevent overfull lines
\providecommand{\tightlist}{%
  \setlength{\itemsep}{0pt}\setlength{\parskip}{0pt}}
\setcounter{secnumdepth}{5}
\ifxetex
  % Load polyglossia as late as possible: uses bidi with RTL langages (e.g. Hebrew, Arabic)
  \usepackage{polyglossia}
  \setmainlanguage[]{english}
\else
  \usepackage[shorthands=off,main=english]{babel}
\fi

%Define cslreferences environment, required by pandoc 2.8
%https://github.com/rstudio/rmarkdown/issues/1649


\providecommand{\tightlist}{%
  \setlength{\itemsep}{0pt}\setlength{\parskip}{0pt}}


\date{}
\newcommand{\trTitle}{Proposed decision tables for copper rockfish (\emph{Sebastes caurinus}) off the U.S. West Coast}
\newcommand{\trYear}{2021}
\newcommand{\trMonth}{September}
\newcommand{\trAuthsLong}{true}
\newcommand{\trAuthsBack}{Wetzel, C.R}
\newcommand{\trCitation}{
\begin{hangparas}{1em}{1}
\trAuthsBack{}. \trYear{}. \trTitle{}. \glsentrylong{pfmc}, Portland, Oregon. \pageref{LastPage}{}\,p.
\end{hangparas}}

\AtBeginDocument{\tagstructbegin{tag=Document}}
\AtEndDocument{\tagstructend}
\pretocmd{\maketitle}{\tagstructbegin{tag=H1}\tagmcbegin{tag=H1}}{}{}
\apptocmd{\maketitle}{\tagmcend\tagstructend}{}{}

\begin{document}

%%%%% Frontmatter %%%%%

% Footnote symbols in front matter
\renewcommand*{\thefootnote}{\fnsymbol{footnote}}

\small
\thispagestyle{empty}
\pagenumbering{roman}
\noindent
\begin{center}
\title{Proposed decision tables for copper rockfish (\emph{Sebastes caurinus}) off the U.S. West Coast}
% \textnormal{\MakeTextUppercase{\trTitle{}}}
\vspace{1.5cm}
{\Large\textbf\newline{Proposed decision tables for copper rockfish (\emph{Sebastes caurinus}) off the U.S. West Coast}}
\vfill
by\\
Chantel R. Wetzel\textsuperscript{1}\vfill
\textsuperscript{1}Northwest Fisheries Science Center, U.S. Department of Commerce, National Oceanic and Atmospheric Administration, National Marine Fisheries Service, 2725 Montlake Boulevard East, Seattle, Washington 98112\vfill
\trMonth{} \trYear{}
\end{center}
\clearpage

% Fourth page: Colophon
\thispagestyle{empty}
\vspace*{\fill}
\begin{center}
\copyright{} \glsentrylong{pfmc}, \trYear{}\\
\end{center}
\par
\bigskip
\noindent
Correct citation for this publication:
\bigskip
\par
\trCitation{}
\clearpage

% Add TOC to pdf bookmarks (clickable pdf)
\pdfbookmark[1]{\contentsname}{toc}

% Table of contents page, lists of figures and tables
\tableofcontents\clearpage
\label{TRlastRoman}
\clearpage

% Table of contents
\newpage
\thispagestyle{empty} % to remove page number

% Settings for the main document
\pagenumbering{arabic}  % Regular page numbers
\pagestyle{plain}  % No page number on first page of main document, use 'empty'
\renewcommand*{\thefootnote}{\arabic{footnote}}  % Back to numeric footnotes
\setcounter{footnote}{0}  % And start at 1
\renewcommand{\headrulewidth}{0.5pt}
\renewcommand{\footrulewidth}{0.5pt}
%\pagestyle{fancy}\fancyhead[c]{Draft: Do not cite or circulate}

\newcommand{\lt}{\ensuremath <}
\newcommand{\gt}{\ensuremath >}

\pagebreak
\pagenumbering{roman}
\setcounter{page}{1}

\renewcommand{\thetable}{\roman{table}}
\renewcommand{\thefigure}{\roman{figure}}

\setlength\parskip{0.5em plus 0.1em minus 0.2em}

\pagebreak
\setlength{\parskip}{5mm plus1mm minus1mm}
\pagenumbering{arabic}
\setcounter{page}{1}
\renewcommand{\thefigure}{\arabic{figure}}
\renewcommand{\thetable}{\arabic{table}}
\setcounter{table}{0}
\setcounter{figure}{0}

\setlength\parskip{0.2em plus 0.1em minus 0.2em}

\tagstructbegin{tag=H1}\tagmcbegin{tag=H1}

\hypertarget{washington}{%
\section{Washington}\label{washington}}

\leavevmode\tagmcend\tagstructend

\tagstructbegin{tag=P}\tagmcbegin{tag=P}

The axes of uncertainty in the decision table is based on the uncertainty around the spawning biomass in 2021 via the log({\tagstructbegin{tag=Formula}\tagmcbegin{tag=Formula}\(R0\)\leavevmode\tagmcend\tagstructend}) parameter. The within model uncertainty estimated from the model was low ({\tagstructbegin{tag=Formula}\tagmcbegin{tag=Formula}\(\sigma\)\leavevmode\tagmcend\tagstructend} = 0.10) which resulted in little variation in the low and high state of nature relative to the base model if used. Meanwhile, the default category 2 {\tagstructbegin{tag=Formula}\tagmcbegin{tag=Formula}\(\sigma\)\leavevmode\tagmcend\tagstructend} of 1.0 resulted in a very large range across the states of nature. As an alternative approach to determine a low and high state of nature from the base model the model uncertanties across the three other area based copper rockfish assessments were averaged to arise at a a {\tagstructbegin{tag=Formula}\tagmcbegin{tag=Formula}\(\sigma\)\leavevmode\tagmcend\tagstructend} = 0.35. of This {\tagstructbegin{tag=Formula}\tagmcbegin{tag=Formula}\(\sigma\)\leavevmode\tagmcend\tagstructend} value was used to identify the 12.5 and 87.5 percentiles of the asymptotic standard deviation for the current year, 2021, spawning biomass from the base model to identify the low and high states of nature (i.e., 1.15 standard deviations corresponding to the 12.5 and 87.5 percentiles). Once the 2021 spawning biomass for the low and high states of nature were identified a search across log({\tagstructbegin{tag=Formula}\tagmcbegin{tag=Formula}\(R0\)\leavevmode\tagmcend\tagstructend}) values were done to attain the current year spawning biomass values. The log({\tagstructbegin{tag=Formula}\tagmcbegin{tag=Formula}\(R0\)\leavevmode\tagmcend\tagstructend}) values that corresponded with the lower and upper percentiles were 1.89 and 2.12.

\leavevmode\tagmcend\tagstructend\par

\tagstructbegin{tag=P}\tagmcbegin{tag=P}

Across the low and high states of nature and across alternative future harvest scenarios the fraction of unfished ranges between 0.37 - 0.58 by the end of the 10 year projection period (Table \ref{tab:wa-dec-tab}). The fraction unfished under the low state of nature declines below that management target by the end of the projection period.

\leavevmode\tagmcend\tagstructend\par

\begingroup\fontsize{10}{12}\selectfont
\begingroup\fontsize{10}{12}\selectfont

\begin{longtable}[t]{l>{\raggedright\arraybackslash}p{0.8cm}>{\raggedright\arraybackslash}p{0.8cm}>{\raggedright\arraybackslash}p{1.45cm}>{\raggedright\arraybackslash}p{1.45cm}>{\raggedright\arraybackslash}p{1.45cm}>{\raggedright\arraybackslash}p{1.45cm}>{\raggedright\arraybackslash}p{1.45cm}>{\raggedright\arraybackslash}p{1.45cm}}
\caption{\label{tab:wa-dec-tab}Decision table summary of 10 year projections beginning in 2023 for alternative states of nature based on an axis of uncertainty around initial stock size. Columns range over low, mid, and high states of nature and rows range over different catch level assumptions.}\\
\toprule
\multicolumn{3}{c}{ } & \multicolumn{2}{c}{log(R0)=1.89} & \multicolumn{2}{c}{log(R0)=2.03} & \multicolumn{2}{c}{log(R0)=2.21} \\
\cmidrule(l{3pt}r{3pt}){4-5} \cmidrule(l{3pt}r{3pt}){6-7} \cmidrule(l{3pt}r{3pt}){8-9}
  & Year & Catch & Spawning Biomass & Fraction Unfished & Spawning Biomass & Fraction Unfished & Spawning Biomass & Fraction Unfished\\
\hline
	&	2021	&	2.11	&	2.15	&	0.324	&	3.20	&	0.419	&	4.75	&	0.519	\\
&	2022	&	2.10	&	2.16	&	0.324	&	3.22	&	0.421	&	4.79	&	0.522	\\
&	2023	&	1.88	&	2.17	&	0.326	&	3.25	&	0.424	&	4.82	&	0.526	\\
&	2024	&	1.89	&	2.20	&	0.330	&	3.29	&	0.431	&	4.88	&	0.533	\\
ACL	&	2025	&	1.89	&	2.23	&	0.335	&	3.34	&	0.437	&	4.94	&	0.539	\\
P*=0.45	&	2026	&	1.89	&	2.26	&	0.340	&	3.39	&	0.443	&	5.00	&	0.545	\\
&	2027	&	1.90	&	2.29	&	0.344	&	3.43	&	0.448	&	5.05	&	0.551	\\
&	2028	&	1.90	&	2.32	&	0.348	&	3.47	&	0.454	&	5.10	&	0.556	\\
&	2029	&	1.90	&	2.34	&	0.352	&	3.51	&	0.459	&	5.14	&	0.561	\\
&	2030	&	1.90	&	2.37	&	0.356	&	3.54	&	0.463	&	5.19	&	0.566	\\
&	2031	&	1.89	&	2.39	&	0.360	&	3.58	&	0.468	&	5.23	&	0.570	\\
&	2032	&	1.89	&	2.42	&	0.364	&	3.61	&	0.472	&	5.27	&	0.575	\\*
 \hline
\end{longtable}
\endgroup{}
\endgroup{}


\clearpage

\tagstructbegin{tag=H1}\tagmcbegin{tag=H1}

\hypertarget{oregon}{%
\section{Oregon}\label{oregon}}

\leavevmode\tagmcend\tagstructend

\tagstructbegin{tag=P}\tagmcbegin{tag=P}

The axes of uncertainty in the decision table is based on the uncertainty around the spawning biomass in 2021 ({\tagstructbegin{tag=Formula}\tagmcbegin{tag=Formula}\(\sigma\)\leavevmode\tagmcend\tagstructend} = 0.42 ) via natural mortality parameter. The {\tagstructbegin{tag=Formula}\tagmcbegin{tag=Formula}\(\sigma\)\leavevmode\tagmcend\tagstructend} value was used to identify the 12.5 and 87.5 percentiles of the asymptotic standard deviation for the current year, 2021, spawning biomass from the base model to identify the low and high states of nature (i.e., 1.15 standard deviations corresponding to the 12.5 and 87.5 percentiles). Once the 2021 spawning biomass for the low and high states of nature were identified a search across natural mortality values were done to attain the current year spawning biomass values. The natural mortality values that corresponded with the lower and upper percentiles were 0.096 and 0.116\textsuperscript{-y}.

\leavevmode\tagmcend\tagstructend\par

\tagstructbegin{tag=P}\tagmcbegin{tag=P}

Across the low and high states of nature and across alternative future harvest scenarios the fraction of unfished ranges between 0.36 - 0.70 by the end of the 10 year projection period (Table \ref{tab:or-dec-tab}). The fraction unfished under the low state of nature declines below that management target by the end of the projection period.

\leavevmode\tagmcend\tagstructend\par

\begingroup\fontsize{10}{12}\selectfont
\begingroup\fontsize{10}{12}\selectfont

\begin{longtable}[t]{l>{\raggedright\arraybackslash}p{0.8cm}>{\raggedright\arraybackslash}p{0.8cm}>{\raggedright\arraybackslash}p{1.45cm}>{\raggedright\arraybackslash}p{1.45cm}>{\raggedright\arraybackslash}p{1.45cm}>{\raggedright\arraybackslash}p{1.45cm}>{\raggedright\arraybackslash}p{1.45cm}>{\raggedright\arraybackslash}p{1.45cm}}
\caption{\label{tab:or-dec-tab}Decision table summary of 10 year projections beginning in 2023 for alternative states of nature based on an axis of uncertainty around initial stock size. Columns range over low, mid, and high states of nature and rows range over different catch level assumptions.}\\
\toprule
\multicolumn{3}{c}{ } & \multicolumn{2}{c}{M = 0.096} & \multicolumn{2}{c}{M = 0.108} & \multicolumn{2}{c}{M = 0.116} \\
\cmidrule(l{3pt}r{3pt}){4-5} \cmidrule(l{3pt}r{3pt}){6-7} \cmidrule(l{3pt}r{3pt}){8-9}
  & Year & Catch & Spawning Biomass & Fraction Unfished & Spawning Biomass & Fraction Unfished & Spawning Biomass & Fraction Unfished\\
\hline
	&	2021	&	10.96	&	17.60	&	0.609	&	28.51	&	0.736	&	41.28	&	0.811	\\
&	2022	&	10.96	&	17.02	&	0.589	&	27.97	&	0.722	&	40.78	&	0.801	\\
&	2023	&	15.72	&	16.47	&	0.570	&	27.47	&	0.709	&	40.30	&	0.792	\\
&	2024	&	15.03	&	15.44	&	0.535	&	26.49	&	0.684	&	39.35	&	0.773	\\
ACL	&	2025	&	14.44	&	14.51	&	0.503	&	25.63	&	0.661	&	38.52	&	0.757	\\
P*=0.45	&	2026	&	13.93	&	13.69	&	0.474	&	24.88	&	0.642	&	37.81	&	0.743	\\
&	2027	&	13.47	&	12.97	&	0.449	&	24.23	&	0.625	&	37.21	&	0.731	\\
&	2028	&	13.07	&	12.33	&	0.427	&	23.68	&	0.611	&	36.70	&	0.721	\\
&	2029	&	12.74	&	11.77	&	0.408	&	23.21	&	0.599	&	36.28	&	0.713	\\
&	2030	&	12.42	&	11.28	&	0.391	&	22.81	&	0.589	&	35.93	&	0.706	\\
&	2031	&	12.15	&	10.84	&	0.376	&	22.47	&	0.580	&	35.64	&	0.700	\\
&	2032	&	11.91	&	10.46	&	0.362	&	22.19	&	0.572	&	35.41	&	0.696	\\*


 \hline
\end{longtable}
\endgroup{}
\endgroup{}


\clearpage

\tagstructbegin{tag=H1}\tagmcbegin{tag=H1}

\hypertarget{california-north-of-point-conception}{%
\section{California North of Point Conception}\label{california-north-of-point-conception}}

\leavevmode\tagmcend\tagstructend

\tagstructbegin{tag=P}\tagmcbegin{tag=P}

The axes of uncertainty in the decision table is based on the uncertainty around the spawning biomass in 2021 ({\tagstructbegin{tag=Formula}\tagmcbegin{tag=Formula}\(\sigma\)\leavevmode\tagmcend\tagstructend} = 0.30) via natural mortality parameter. The {\tagstructbegin{tag=Formula}\tagmcbegin{tag=Formula}\(\sigma\)\leavevmode\tagmcend\tagstructend} value was used to identify the 12.5 and 87.5 percentiles of the asymptotic standard deviation for the current year, 2021, spawning biomass from the base model to identify the low and high states of nature (i.e., 1.15 standard deviations corresponding to the 12.5 and 87.5 percentiles). Once the 2021 spawning biomass for the low and high states of nature were identified a search across natural mortality values were done to attain the current year spawning biomass values. The natural mortality values that corresponded with the lower and upper percentiles were 0.098 and 0.123\textsuperscript{-y}.

\leavevmode\tagmcend\tagstructend\par

\tagstructbegin{tag=P}\tagmcbegin{tag=P}

Across the low and high states of nature and across alternative future harvest scenarios the fraction of unfished ranges between 0.31 - 0.60 by the end of the 10 year projection period (Table \ref{tab:nca-dec-tab}). The fraction unfished across the state of natures assuming the full ABC removals all increase over the projection period from the fraction unfished in 2023.

\leavevmode\tagmcend\tagstructend\par

\begingroup\fontsize{10}{12}\selectfont
\begingroup\fontsize{10}{12}\selectfont

\begin{longtable}[t]{l>{\raggedright\arraybackslash}p{0.8cm}>{\raggedright\arraybackslash}p{0.8cm}>{\raggedright\arraybackslash}p{1.45cm}>{\raggedright\arraybackslash}p{1.45cm}>{\raggedright\arraybackslash}p{1.45cm}>{\raggedright\arraybackslash}p{1.45cm}>{\raggedright\arraybackslash}p{1.45cm}>{\raggedright\arraybackslash}p{1.45cm}}
\caption{\label{tab:nca-dec-tab}Decision table summary of 10 year projections beginning in 2023 for alternative states of nature based on an axis of uncertainty around initial stock size. Columns range over low, mid, and high states of nature and rows range over different catch level assumptions.}\\
\toprule
\multicolumn{3}{c}{ } & \multicolumn{2}{c}{M = 0.098} & \multicolumn{2}{c}{M = 0.108} & \multicolumn{2}{c}{M = 0.123} \\
\cmidrule(l{3pt}r{3pt}){4-5} \cmidrule(l{3pt}r{3pt}){6-7} \cmidrule(l{3pt}r{3pt}){8-9}
  & Year & Catch & Spawning Biomass & Fraction Unfished & Spawning Biomass & Fraction Unfished & Spawning Biomass & Fraction Unfished\\
\hline
	&	2021	&	115.60	&	115.88	&	0.259	&	163.51	&	0.393	&	230.01	&	0.607	\\
&	2022	&	113.10	&	112.53	&	0.251	&	161.34	&	0.388	&	227.55	&	0.601	\\
&	2023	&	78.76	&	109.35	&	0.244	&	158.68	&	0.382	&	223.59	&	0.590	\\
&	2024	&	79.57	&	110.52	&	0.247	&	160.12	&	0.385	&	222.98	&	0.589	\\
ACL	&	2025	&	80.59	&	112.87	&	0.252	&	162.35	&	0.390	&	222.84	&	0.588	\\
P*=0.45	&	2026	&	81.67	&	116.08	&	0.259	&	165.13	&	0.397	&	223.14	&	0.589	\\
&	2027	&	82.42	&	119.75	&	0.268	&	168.18	&	0.404	&	223.69	&	0.591	\\
&	2028	&	82.93	&	123.60	&	0.276	&	171.31	&	0.412	&	224.37	&	0.592	\\
&	2029	&	83.48	&	127.51	&	0.285	&	174.43	&	0.419	&	225.09	&	0.594	\\
&	2030	&	83.86	&	131.39	&	0.294	&	177.46	&	0.427	&	225.79	&	0.596	\\
&	2031	&	84.17	&	135.22	&	0.302	&	180.39	&	0.434	&	226.45	&	0.598	\\
&	2032	&	84.51	&	139.02	&	0.311	&	183.21	&	0.441	&	227.07	&	0.600	\\*

 \hline
\end{longtable}
\endgroup{}
\endgroup{}


\clearpage

\tagstructbegin{tag=H1}\tagmcbegin{tag=H1}

\hypertarget{california-south-of-point-conception}{%
\section{California South of Point Conception}\label{california-south-of-point-conception}}

\leavevmode\tagmcend\tagstructend

\tagstructbegin{tag=P}\tagmcbegin{tag=P}

The axes of uncertainty in the decision table is based on the uncertainty around the spawning biomass in 2021 ({\tagstructbegin{tag=Formula}\tagmcbegin{tag=Formula}\(\sigma\)\leavevmode\tagmcend\tagstructend} = 0.33 ) via the log({\tagstructbegin{tag=Formula}\tagmcbegin{tag=Formula}\(R0\)\leavevmode\tagmcend\tagstructend}) parameter. The {\tagstructbegin{tag=Formula}\tagmcbegin{tag=Formula}\(\sigma\)\leavevmode\tagmcend\tagstructend} value was used to identify the 12.5 and 87.5 percentiles of the asymptotic standard deviation for the current year, 2021, spawning biomass from the base model to identify the low and high states of nature (i.e., 1.15 standard deviations corresponding to the 12.5 and 87.5 percentiles). Once the 2021 spawning biomass for the low and high states of nature were identified a search across log({\tagstructbegin{tag=Formula}\tagmcbegin{tag=Formula}\(R0\)\leavevmode\tagmcend\tagstructend}) values were done to attain the current year spawning biomass values. The log({\tagstructbegin{tag=Formula}\tagmcbegin{tag=Formula}\(R0\)\leavevmode\tagmcend\tagstructend}) values that corresponded with the lower and upper percentiles were 5.44 and 5.55.

\leavevmode\tagmcend\tagstructend\par

\tagstructbegin{tag=P}\tagmcbegin{tag=P}

Across the low and high states of nature and across alternative future harvest scenarios the fraction of unfished ranges between 0.21 - 0.39 by the end of the 10 year projection period (Table \ref{tab:sca-dec-tab}). The fraction unfished across the state of natures assuming the ACL removals (the ABC adjusted by the 40:10 harvest control rule) remains below the management target.

\leavevmode\tagmcend\tagstructend\par

\begingroup\fontsize{10}{12}\selectfont
\begingroup\fontsize{10}{12}\selectfont

\begin{longtable}[t]{l>{\raggedright\arraybackslash}p{0.8cm}>{\raggedright\arraybackslash}p{0.8cm}>{\raggedright\arraybackslash}p{1.45cm}>{\raggedright\arraybackslash}p{1.45cm}>{\raggedright\arraybackslash}p{1.45cm}>{\raggedright\arraybackslash}p{1.45cm}>{\raggedright\arraybackslash}p{1.45cm}>{\raggedright\arraybackslash}p{1.45cm}}
\caption{\label{tab:sca-dec-tab}Decision table summary of 10 year projections beginning in 2023 for alternative states of nature based on an axis of uncertainty around initial stock size. Columns range over low, mid, and high states of nature and rows range over different catch level assumptions.}\\
\toprule
\multicolumn{3}{c}{ } & \multicolumn{2}{c}{log(R0)=5.44} & \multicolumn{2}{c}{log(R0)=5.50} & \multicolumn{2}{c}{log(R0)=5.55} \\
\cmidrule(l{3pt}r{3pt}){4-5} \cmidrule(l{3pt}r{3pt}){6-7} \cmidrule(l{3pt}r{3pt}){8-9}
  & Year & Catch & Spawning Biomass & Fraction Unfished & Spawning Biomass & Fraction Unfished & Spawning Biomass & Fraction Unfished\\
\hline
	&	2021	&	90.80	&	28.79	&	0.130	&	42.28	&	0.181	&	62.04	&	0.253	\\
&	2022	&	88.90	&	25.30	&	0.114	&	38.97	&	0.167	&	58.74	&	0.240	\\
&	2023	&	8.79	&	21.29	&	0.096	&	35.28	&	0.151	&	55.19	&	0.225	\\
&	2024	&	11.44	&	23.29	&	0.105	&	37.92	&	0.163	&	58.32	&	0.238	\\
ACL	&	2025	&	14.58	&	26.06	&	0.118	&	41.44	&	0.178	&	62.38	&	0.255	\\
P*=0.45	&	2026	&	17.74	&	29.41	&	0.133	&	45.68	&	0.196	&	67.16	&	0.274	\\
&	2027	&	20.56	&	32.94	&	0.149	&	50.24	&	0.216	&	72.26	&	0.295	\\
&	2028	&	22.97	&	36.31	&	0.164	&	54.75	&	0.235	&	77.34	&	0.316	\\
&	2029	&	25.08	&	39.44	&	0.178	&	59.07	&	0.253	&	82.19	&	0.336	\\
&	2030	&	26.91	&	42.36	&	0.192	&	63.17	&	0.271	&	86.75	&	0.354	\\
&	2031	&	28.57	&	45.15	&	0.204	&	67.06	&	0.288	&	90.99	&	0.372	\\
&	2032	&	30.10	&	47.85	&	0.216	&	70.78	&	0.304	&	94.92	&	0.388	\\*
 \hline
\end{longtable}
\endgroup{}
\endgroup{}


\clearpage
\end{document}
